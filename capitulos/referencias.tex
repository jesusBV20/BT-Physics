
% Bibliografía -
\addcontentsline{toc}{section}{Referencias} \normalsize

\bibliography{ML}

\begin{thebibliography}{9}

\bibitem{STEM_basics}
Peter D. Nellist (2011). \textit{The Principles of STEM Imaging.} S.J. Pennycook, P.D. Nellist (Ed.). \textit{Scanning Transmission Electron Microscopy }(pp. 91-92). DOI 10.1007/978-1-4419-7200-2\_2. 

\bibitem{tesis_gabriel}
Gabriel Sánchez Santolino (2015). \textit{Advanced electron microscopy characterization of complex oxide interfaces}. Tesis, Universidad Complutense de Madrid, pp. 9-30.

\bibitem{foto_intro}
J.M. Thomas et al. (2015). \textit{The rapidly changing face of electron microscopy}. Chemical Physics Letters 631–632, 103–113.

\bibitem{maria}
M. Varela et al. (2012). \textit{Scanning transmission electron microscopy of oxides}. Tsymbal, Evgeny Y. et al. (Ed.) \textit{Multifunctional Oxide Heterostructures} (pp. 123-156).

\bibitem{e_gun}
Shao and Khursheed, \textit{A Review Paper on “Graphene Field Emission for Electron Microscopy} / Appl. Sci. 2018, 8, 868.

\bibitem{databook}
S. L. Brunton and J. N. Kutz (2017). \textit{Data-Driven Science and Engineering: Machine Learning, Dynamical Systems, and Control}.

\bibitem{datos}
Abruña, Héctor; A Muller, David; Xu, Rui; Ophus, Colin; Miao, Jianwei; Hovden, Robert; et al. (2016): \textit{Nanomaterial datasets to advance tomography in scanning transmission electron microscopy}. figshare. Collection. \href{https://dx.doi.org/10.6084/m9.figshare.c.2185342}{https://dx.doi.org/10.6084/m9.figshare.c.2185342}

\bibitem{repo}
Jesús Bautista Villar (2022). \textit{STEM methods}. Repositorio, GitHub. \\ \url{https://github.com/jesusBV20/STEM_methods}.

\bibitem{biblia}
Horn, Roger A.; Johnson, Charles R. (1985). \textit{Matrix Analysis. Cambridge University Press}. ISBN 978-0-521-38632-6.

% \bibitem{comparacion}
% N. Naheed et al. \textit{Comparison of SVD and FFT in image compression}. 2015 \textit{International Conference on Computational Science and Computational Intelligence}, pp. 526-530.

\bibitem{matlab_circulos}
T.J. Atherton, D.J. Kerbyson (1999). \textit{Image and Vision Computing}, 17, 795–803.

\bibitem{ml}
Jiadong D., Xiaoxu Z., Stephen J.P. (2019). \textit{A machine perspective of atomic defects in scanning transmission electron microscopy}. InfoMat: Vol 1, No 3, 359-375.

\bibitem{EELS_PCA}
Potapov, P., Lubk, A. (2019). \textit{Optimal principal component analysis of STEM XEDS spectrum images}. \textit{Adv Struct Chem Imag} 5, 4. 

\bibitem{CNN}
Krizhevsky A, et. al. \textit{ImageNet classification
with deep convolutional neural networks}. In: Pereira F, et al. eds. \textit{Advances in Neural
Information Processing Systems 25}. Red Hook, NY: Curran
Associates, Inc; 2012:1097-1105.

\bibitem{red}
Arroyo-Hernández, et al. Sistema de detección y clasificación automática de granos de polen mediante técnicas de procesado digital de imágenes. Uniciencia. 27. 59-73. 

\bibitem{STO}
Kalabukhov, A., et al. (2007). \textit{Effect of oxygen vacancies in the SrTiO$_3$ substrate on the electrical properties of the LaAlO$_3/$SrTiO$_3$ interface}. Physical Review B, 75(12).

\bibitem{paquito}
Francisco Fernández Cañizares (2021). \textit{Advanced microscopy techniques applied to cutting edge material systems}. Trabajo fin de Grado, Universidad Complutense de Madrid.

\end{thebibliography}


