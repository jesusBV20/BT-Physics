% Portada --

\begin{titlepage}
\centering
{ \bfseries \Large UNIVERSIDAD COMPLUTENSE DE MADRID}
\vspace{0.5cm}

{\bfseries  \Large FACULTAD DE CIENCIAS FÍSICAS} 
\vspace{1cm}

{\large DEPARTAMENTO DE FÍSICA DE MATERIALES}
\vspace{0.8cm}

{\includegraphics[width=0.35\textwidth]{fig/logo_UCM.png}}
\vspace{0.8cm}

{\bfseries \Large TRABAJO DE FIN DE GRADO}

\vfill
{\Large Código de TFG:  FM17 } \vspace{5mm} \\
{\Large Técnicas avanzadas de microscopía aplicadas al estudio \\ de nuevos materiales}\vspace{5mm} \\
{\Large Advanced microscopy techniques applied to cuttingedge \\ material systems}\vspace{5mm} \\
{\Large Supervisor/es: María Varela del Arco}\\ 
\vfill

{\bfseries \LARGE Jesús Bautista Villar} \\
\noindent\rule{8cm}{0.4pt}\vspace{5mm}

{\large Grado en Física}\vspace{2.5mm} \\
{\large Curso acad\'emico 2021-2022}\vspace{2.5mm} \\
{\large Convocatoria Junio} \\ 

\end{titlepage}

% Abstract --
%Nota: el título extendido (si procede), el resumen y el abstract deben estar en una misma página y su extensión no debe superar una página. Tamaño mínimo 11 pto.

\newpage
\thispagestyle{empty} % Página flotante para quitar numeración

{\bfseries \large Resumen:} \vspace{5mm}

En todos los campos de la ciencia resulta vital disponer de dispositivos de caracterización que nos permitan medir y experimentar con resultados teóricos. Es por esta razón que, como consecuencia de un constante avance científico y tecnológico, en los últimos han surgido técnicas de microscopía cada vez más punteras. En este trabajo nos centramos en la STEM de aberración corregida, una herramienta fundamental en numerosos campos de la física que recientemente ha ganado un gran protagonismo, debido a su gran resolución y a la amplia diversidad de datos que puede extraer de una única muestra. Para digerir tal cantidad de información, se hace necesaria la introducción de métodos estadísticos y algoritmos de inteligencia artificial. El principal objetivo de este proyecto será explorar la relevancia y régimen de aplicabilidad de dichas herramientas de análisis en imágenes 4D-STEM.

\vspace{1cm}

{\bfseries \large Abstract: }\vspace{5mm} 

Characterization techniques form the basis for practical knowledge in every field of science. Accordingly, the huge advances in science and technology enable the raise of brand new microscopy techniques. This project focuses on aberration-corrected scanning transmission electron microscopy (STEM) because of its great atomic-resolution and its wealth of sophisticated multidimensional information. Nevertheless, it's required to introduce advanced statistical methods and artificial intelligent algorithms to deal with this high volume of data. In consequence, the principal objective of this project is to review the main 4-Dimensional Scanning Transmission Electron Microscopy (4D-STEM) data analysis techniques.

\vspace{1cm}

