% Portada --

\begin{titlepage}
\centering
{ \bfseries \Large UNIVERSIDAD COMPLUTENSE DE MADRID}
\vspace{0.5cm}

{\bfseries  \Large FACULTAD DE CIENCIAS FÍSICAS} 
\vspace{1cm}

{\large DEPARTAMENTO DE FÍSICA DE MATERIALES}
\vspace{0.8cm}

{\includegraphics[width=0.35\textwidth]{fig/logo_UCM.png}}
\vspace{0.8cm}

{\bfseries \Large TRABAJO DE FIN DE GRADO}

\vfill
{\Large Código de TFG:  FM17 } \vspace{5mm} \\
{\Large Técnicas avanzadas de microscopía aplicadas al estudio \\ de nuevos materiales}\vspace{5mm} \\
{\Large Advanced microscopy techniques applied to cutting edge \\ material systems}\vspace{5mm} \\
{\Large Supervisor/es: María Varela del Arco}\\ 
\vfill

{\bfseries \LARGE Jesús Bautista Villar} \\
\noindent\rule{8cm}{0.4pt}\vspace{5mm}

{\large Grado en Física}\vspace{2.5mm} \\
{\large Curso acad\'emico 2021-2022}\vspace{2.5mm} \\
{\large Convocatoria Junio} \\ 

\end{titlepage}

% Abstract --
%Nota: el título extendido (si procede), el resumen y el abstract deben estar en una misma página y su extensión no debe superar una página. Tamaño mínimo 11 pto.

\newpage
\thispagestyle{empty} % Página flotante para quitar numeración

{\bfseries \large Resumen:} \vspace{5mm}

En todos los campos de la ciencia resulta vital disponer de dispositivos de caracterización que nos permitan medir y experimentar con resultados teóricos. Es por esta razón que, como consecuencia de un constante avance científico y tecnológico, en el contexto de la ciencia de materiales han surgido en los últimos años técnicas de microscopía cada vez más punteras. En este trabajo nos centramos en la microscopía electrónica de transmisión con barrido (STEM) de aberración corregida, una herramienta fundamental en numerosos campos de la física que recientemente ha ganado un gran protagonismo, debido a su gran resolución y a la amplia diversidad de datos que puede extraer de una única muestra. Para el enorme volumen de datos resultante, se hace necesaria la introducción de métodos estadísticos y algoritmos de inteligencia artificial. El principal objetivo de este proyecto consistirá precisamente en explorar la relevancia y régimen de aplicabilidad de dichas herramientas de análisis en microscopía STEM multidimensional, o 4D-STEM. Estas técnicas han adquirido enorme popularidad en los últimos años debido a que permiten estudiar campos electromagnéticos en materiales con resolución atómica. Sin duda, los métodos de 4D-STEM resultarán cruciales en el futuro inmediato para la identificación y estudio de materiales de enorme interés en campos tan variopintos como la energía, la espintrónica o la nanotecnología.

\vspace{1cm}

{\bfseries \large Abstract: }\vspace{5mm} 

All branches of science need of techniques capable of measuring the relevant behaviors and comparing them to theoretical predictions in order to harness the phenomena at hand, and Materials Science is second to none in this respect. In the last couple of decades, in fact, we have witnessed dramatic advances in electron microscopy techniques capable of probing matter at the atomic scale in real space. However, more often than not, technical advances have come hand in hand with the production of large volumes of experimental data, which can be difficult to analyze and hamper progress. The main objective of this project is, precisely, to review the latest advances and achievements of artificial intelligence techniques applied to the quantification of scanning transmission electron microscopy (STEM) datasets. Special attention will be paid to applications to diffraction, multidimensional 4D-STEM images. These have become very popular in the last years because they allow mapping of electromagnetic fields in materials with atomic resolution, paving the way to new developments and materials systems of interest in fields as diverse as energy, spintronics or nanotechnology.

\vspace{1cm}

