
\usepackage[spanish]{babel} %Español 
\usepackage[utf8]{inputenc} %Para poder poner tildes
\usepackage{vmargin} %Para modificar los márgenes
\setmargins{2.5cm}{1.5cm}{16.5cm}{23.42cm}{10pt}{1cm}{0pt}{2cm}
% Margen izquierdo, superior, anchura del texto, altura del texto, altura de los encabezados, espacio entre el texto y los encabezados, altura del pie de página, espacio entre el texto y el pie de página.

% Figuras, imágenes y sus captions
\usepackage{graphicx} 
\usepackage[font=small,labelfont=bf]{caption}

% Otros paquetes
\usepackage{floatpag}
\usepackage{ dsfont }
\usepackage{amsmath } %Matrices
\usepackage{hyperref} %\autoref{}
\usepackage{wrapfig} % Texto y figura "side by side"

\hypersetup{
    colorlinks=true,
    linkcolor=black,
    filecolor=magenta,      
    urlcolor=cyan,
    pdftitle={Overleaf Example},
    pdfpagemode=FullScreen,
    }
\urlstyle{same}

% Macros
\setlength{\parindent}{0cm} % Default is 15pt.
% Para añadir indentación manualmente: \hangindent=0.7cm 

\renewcommand{\baselinestretch}{1} % Interlineado